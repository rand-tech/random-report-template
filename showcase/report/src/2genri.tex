% Point: 用いた物理的原理を数式を交えて書く
(中略)

\subsection*{便利Packageの紹介}
\subsubsection{\texttt{siunitx}}
\texttt{siunitx}を使うことで,手間を減らせます.
例えば,\verb|\SI{1}{m}, \SI{5}{\celsius}, \SI{10}{kg.m.s^{-2}}|が\SI{1}{m}, \SI{5}{\celsius}, \SI{10}{kg.m.s^{-2}}のようになります.

\subsubsection{\texttt{Physics} package}
\texttt{physics} package\footnote{参照: \url{https://mirrors.ibiblio.org/CTAN/macros/latex/contrib/physics/physics.pdf}}を使うことで,簡潔な表現ができる場合があります.例えば,
\verb|\frac{\mathrm{d}}{\mathrm{d}x}|は\verb|\dv{x}|と同等な結果をもたらします.
\verb|\dv{x}f = \dv{f}{x}|は次のように表示されます.
\begin{align*}
  \dv{x}f = \dv{f}{x}
\end{align*}
